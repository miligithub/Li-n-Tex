\documentclass[12pt]{mines-thesis}
% Font Size: 10-12 point type

%========================================
%                              Set-up                              
%========================================
% Text will be double spaced or 1.5 spaced. 
%\OnehalfSpacing 
\DoubleSpacing


%========================================
%                           Packages                             
%========================================
\usepackage{lipsum} % dummy text


\begin{document}
	%========================================
	%                            Front Matter                              
	%========================================
	\autotitle %automatically change the title into upper cases and reshape it to an inverted pyramid
	\title{Thesis title centered on the page vertically and horizontally, in all upper case letters
		and in an inverted pyramid shape. Math mode: $\frac{2^{15}}{\pi}$
	}
	% If you want to break the title by yourself, you must use \protect \\  or add an empty line.
	% For example,
	%\title{
	%	Thesis title centered on the page vertically and horizontally, \protect\\
	%	%	
	%	in all upper case letters and in an inverted 
	%	
	%	pyramid shape. Math mode: $\frac{2^{15}}{\pi}$
	%}
	
	\author{Student Name}  % this is your name exactly as you want it
	\year{2020}     % this is the year of defence
	\degree{Doctor of Philosophy}{Computer Science} %
	\advisor{Dr. Thesis Advisor}
	%\coadvisor{Dr. Co-Advisor} 
	\department{Department of XXX}
	\departmenthead{Dr. Department Head}
	
	
	%==================Abstracts===============
	% 1. Are generally 200-300 words in length, 
	% 2. Consist of one to two paragraphs of information,
	% 3. Does not usually contain citations, 
	% 4. Do not repeat the thesis title, and
	% 5. Each paragraph should be indented.
	
	\begin{abstract}
		The abstract is a concise, one to three sentence statement of the thesis problem, a brief description consisting of no more than a few sentences describing the research method or design, and a report of the major findings and conclusions.
		%
		
		%
		The abstract submitted online at the time of thesis submission should be the same as the abstract inside the thesis. ProQuest continues to publish print indexes that have maximum abstract lengths of 150 words for Masters and 350 words for PhDs. Abstracts exceeding these limits will be truncated in the print indexes so it may be wise to work within those word limits in most cases.
	\end{abstract}
	

	\makefrontmatter


%========================================
%                              Main Body                              
%========================================
\lipsum[1-20]
%\begin{center}
{
	\centering
	\MakeTextUppercase{
		Thesis title centered on the \protect\\page vertically and horizontally,\protect\\
		%	
		in all upper case letters and in an inverted 
		pyramid shape. Math mode: $\frac{2^{15}}{\pi}$
	}
	
}
%\centering


%\end{center}

\newpage
\invpyr{Thesis title centered on the page vertically and horizontally,\\
	%	
	in all upper case letters and in an inverted 
	
	pyramid shape. Math mode: $\frac{2^{15}}{\pi}$

	
	\MakeTextUppercase{
		Thesis title centered on the \\page vertically and horizontally,\\
		%	
		in all upper case letters and in an inverted 
		pyramid shape. Math mode: $\frac{2^{15}}{\pi}$
	}
	
}  

\clearpage
\vspace*{\fill}
\begin{center}
	\begin{minipage}{\textwidth}
	\invpyr{Thesis title centered on the page vertically and horizontally,\\
		%	
		in all upper case letters and in an inverted 
		
		pyramid shape. Math mode: $\frac{2^{15}}{\pi}$
}  
\end{minipage}
\end{center}
\vspace{\fill}
\begin{center}
	by\\
	author
\end{center}
\clearpage
alibrating the response of a GPR system is essential for making measurements of subsurface materials properties. Duke (1990) calibrated the overall response of a GPR system by making measurements of the . . .
1.1 Background and Previous Work
This chapter describes the methodology that has been used to determine the response of an impulse GPR. The characterization includes a response function for the receiving electronics, simulations . . .
1.2 Signal Processing Tools
There are many techniques for making high frequency electrical measurements in electrical networks and antenna systems, and there are also many methods for manipulating the data from these measurements . . .
1.2.1 Convolution and Deconvolution Methods
Convolution is a mathematical operation that can be used to describe how a linear network element modifies a signal as the signal passes through it . . .
1.2.2 Scattering Parameters
\newpage
alibrating the response of a GPR system is essential for making measurements of subsurface materials properties. Duke (1990) calibrated the overall response of a GPR system by making measurements of the . . .
1.1 Background and Previous Work
This chapter describes the methodology that has been used to determine the response of an impulse GPR. The characterization includes a response function for the receiving electronics, simulations . . .
1.2 Signal Processing Tools
There are many techniques for making high frequency electrical measurements in electrical networks and antenna systems, and there are also many methods for manipulating the data from these measurements . . .
1.2.1 Convolution and Deconvolution Methods
Convolution is a mathematical operation that can be used to describe how a linear network element modifies a signal as the signal passes through it . . .
1.2.2 Scattering Parameters



%========================================
%                             Back Matter                               
%========================================

\end{document}
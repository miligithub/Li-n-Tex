\documentclass[12pt,oneside,showtrims]{mines-thesis}
%========================================
%                              Set-up                              
%========================================
\spacing{1.5}


%========================================
%                            Front Matter                              
%========================================
\title{Thesis title centered on the page vertically and horizontally, in all upper case letters and in an inverted pyramid shape $\frac{2^{15}}{\pi}$}         % this is the full title of your work
\author{Author's Name}  % this is your name exactly as you want it
\year{2020}     % this is the year of defence
%\degree{Doctor of Philosophy}{Geophysical Engineering} %
%\advisor{Dr. Thesis Advisor}
%\coadvisor{Dr. Co-Advisor}
%\department{Department of XXX}
%\departmenthead{Dr. Department Head}

%	\begin{enumerate}
%		\item Title page (Required)
%		\item Copyright page
%		\item Submittal page (Required)
%		\item Abstract (Required)
%		\item Table of Contents (Required)
%		\item Lists (Required if included in thesis)
%		\begin{enumerate}
%			\item List of Figures
%			\item List of Tables
%			\item List of Equations (optional)
%			\item List of Plates
%			\item List of Symbols
%		\end{enumerate}
%		\item Acknowledgments
%		\item Dedication          
%	\end{enumerate}
%==================Abstracts===============
% 1. Are generally 200-300 words in length, 
% 2. Consist of one to two paragraphs of information,
% 3. Does not usually contain citations, 
% 4. Do not repeat the thesis title, and
% 5. Each paragraph should be indented.
%
%\begin{abstract}
%	The abstract is a concise, one to three sentence statement of the thesis problem, a brief description consisting of no more than a few sentences describing the research method or design, and a report of the major findings and conclusions.
%	%
%	
%	%
%	The abstract submitted online at the time of thesis submission should be the same as the abstract inside the thesis. ProQuest continues to publish print indexes that have maximum abstract lengths of 150 words for Masters and 350 words for PhDs. Abstracts exceeding these limits will be truncated in the print indexes so it may be wise to work within those word limits in most cases.
%\end{abstract}

%\usepackage{memsty}
%%%%%%%%%%%%%%%%%%%%%%%%%%%%%
%\usepackage{titlepages}  % code of the example titlepages
%
%\usepackage{memsty}
%%%%%%%%%%%%%%%%%%%%%%%%%%%%%
%\usepackage{titlepages}  % code of the example titlepages
%\usepackage{memlays}     % extra layout diagrams
%\usepackage{dpfloat}     % floats on facing pages
\usepackage{fonttable}[2009/04/01]   % font tables

\begin{document}
	\maketitlepage
	\makecopyrightpage
%	\year
%========================================
%                              Main Body                              
%========================================
alibrating the response of a GPR system is essential for making measurements of subsurface materials properties. Duke (1990) calibrated the overall response of a GPR system by making measurements of the . . .
1.1 Background and Previous Work
This chapter describes the methodology that has been used to determine the response of an impulse GPR. The characterization includes a response function for the receiving electronics, simulations . . .
1.2 Signal Processing Tools
There are many techniques for making high frequency electrical measurements in electrical networks and antenna systems, and there are also many methods for manipulating the data from these measurements . . .
1.2.1 Convolution and Deconvolution Methods
Convolution is a mathematical operation that can be used to describe how a linear network element modifies a signal as the signal passes through it . . .
1.2.2 Scattering Parameters
\newpage
alibrating the response of a GPR system is essential for making measurements of subsurface materials properties. Duke (1990) calibrated the overall response of a GPR system by making measurements of the . . .
1.1 Background and Previous Work
This chapter describes the methodology that has been used to determine the response of an impulse GPR. The characterization includes a response function for the receiving electronics, simulations . . .
1.2 Signal Processing Tools
There are many techniques for making high frequency electrical measurements in electrical networks and antenna systems, and there are also many methods for manipulating the data from these measurements . . .
1.2.1 Convolution and Deconvolution Methods
Convolution is a mathematical operation that can be used to describe how a linear network element modifies a signal as the signal passes through it . . .
1.2.2 Scattering Parameters

%\begin{figure}
%	\centering
%	\begin{showtitle}
%		\titleGM
%	\end{showtitle}
%	\caption{Layout of a title page for a book about books}\label{figure:titleGM}
%\end{figure}
%\begin{figure} \begin{showtitle} \titleJT \end{showtitle} \end{figure}
%\begin{figure} \begin{showtitle} \titleTH \end{showtitle} \end{figure}
%\begin{figure} \begin{showtitle} \titleM \end{showtitle} \end{figure}
%\begin{figure} \begin{showtitle} \titleS \end{showtitle} \end{figure}
%\begin{figure} \begin{showtitle} \titleRF \end{showtitle} \end{figure}
%\begin{figure} \begin{showtitle} \titleDB \end{showtitle} \end{figure}
%\begin{figure} \begin{showtitle} \titleAM \end{showtitle} \end{figure}
%\begin{figure} \begin{showtitle} \titleP \end{showtitle} \end{figure}
%\begin{figure} \begin{showtitle} \titleHL \end{showtitle} \end{figure}
%\begin{figure} \begin{showtitle} \titleVL \end{showtitle} \end{figure}
%\begin{figure} \begin{showtitle} \titleGM \end{showtitle} \end{figure}
%\begin{figure} \begin{showtitle} \titlePM \end{showtitle} \end{figure}
%\begin{figure} \begin{showtitle} \titleAT \end{showtitle} \end{figure}
%\begin{figure} \begin{showtitle} \titleLL \end{showtitle} \end{figure}
%\begin{figure} \begin{showtitle} \titleCC \end{showtitle} \end{figure}
%\begin{figure} \begin{showtitle} \titleDS \end{showtitle} \end{figure}
%\begin{figure} \begin{showtitle} \titleMS \end{showtitle} \end{figure}
%\begin{figure} \begin{showtitle} \titlePW \end{showtitle} \end{figure}
%\begin{figure} \begin{showtitle} \titleSW \end{showtitle} \end{figure}
%\begin{figure} \begin{showtitle} \titleTMB \end{showtitle} \end{figure}
%\begin{figure} \begin{showtitle} \titleJE \end{showtitle} \end{figure}
%\begin{figure} \begin{showtitle} \titleZD \end{showtitle} \end{figure}
%\begin{figure} \begin{showtitle} \titleWH \end{showtitle} \end{figure}
%\begin{figure} \begin{showtitle} \titleBWF \end{showtitle} \end{figure}
%\begin{figure} \begin{showtitle} \titleASUa \end{showtitle} \end{figure}
%\begin{figure} \begin{showtitle} \titleASUt \end{showtitle} \end{figure}
%\begin{figure} \begin{showtitle} \titleRB \end{showtitle} \end{figure}
%\begin{figure} \begin{showtitle} \titleGP \end{showtitle} \end{figure}
%\begin{figure} \begin{showtitle} \titleSI \end{showtitle} \end{figure}
%\begin{figure} \begin{showtitle} \titleBWF \end{showtitle} \end{figure}


%========================================
%                             Back Matter                               
%========================================

\end{document}